% xelatex paper
% xelatex paper && bibtex paper && xelatex paper && xelatex paper

\documentclass{nime-alternate} % Uncomment when publishing final version

% Uncomment only one of the ones below
\usepackage{anonymize} 		   %Uncomment this line to publish
% \usepackage[blind]{anonymize}%Uncomment this line for blind review

\usepackage[utf8]{inputenc}

\begin{document}

% --- Author Metadata here ---
\conferenceinfo{NIME'24,}{4--6 September, Utrecht, The Netherlands.}

\title{Malletwand: the Pendulum as a Handheld Interface to Musical Timing}

\label{key}
\numberofauthors{4}
\author{
\alignauthor
\anonymize{Lyu, S.} \\
  \affaddr{\anonymize{Academy of Arts \& Design, Tsinghua University}}\\
  \affaddr{\anonymize{Beijing, China}}\\
  \email{\anonymize{claire@ayu.land}}
\alignauthor
\anonymize{Li, H.} \\
  \affaddr{\anonymize{Academy of Arts \& Design, Tsinghua University}}\\
  \affaddr{\anonymize{Beijing, China}}\\
  \email{\anonymize{lihanxua22@mails.tsinghua.edu.cn}}
\and
\alignauthor
\anonymize{Wang, R.} \\
  \affaddr{\anonymize{Academy of Arts \& Design, Tsinghua University}}\\
  \affaddr{\anonymize{Beijing, China}}\\
  \email{\anonymize{wangrh22@mails.tsinghua.edu.cn}}
\alignauthor
\anonymize{Mi, H.}\titlenote{\anonymize{Corresponding author.}}\\
  \affaddr{\anonymize{Academy of Arts \& Design, Tsinghua University}}\\
  \affaddr{\anonymize{Beijing, China}}\\
  \email{\anonymize{mhp@tsinghua.edu.cn}}
}
\date{6 February 2024}

\maketitle

\begin{abstract}
We devise and implement an interface in the form of a handheld pendulum device for manipulating the timing of musical playback. The physical properties of the pendulum make this interaction scheme steady and intuitive, and particularly suitable as an introductory means of musical engagement for untrained participants. We build a self-playing glockenspiel around this interface to demonstrate how our design encourages musical exploration and social play. We conclude by discussing potential extrapolations and integrations of the design in mobile and hybrid scenarios.
\end{abstract}
\keywords{Musical interface, pendulums, tactility, pose estimation, HCI}

\ccsdesc[500]{Applied computing~Sound and music computing}
\ccsdesc[300]{Human-centered computing~Interaction devices}
\ccsdesc[300]{Hardware~Sensor applications and deployments}
\printccsdesc


\section{Introduction}

Throughout our experience, we have always been able to capture the instinctive desire to actively and earnestly engage in music from those that have not had the advantage to experience music full-scale.
In our lasting endeavour to expand the reach of musically-meaningful interactive experiences, we set out on a search of an interaction scheme that is easy to understand and operate without musical expertise, while navigating the participant in sensible directions to manipulate the sounds to their own imagination, deepening their connection with the musical elements they have been presented with.

We settled on the periodic motion of the pendulum. The stability and self-recovering property of the pendulum makes for a reliable framework of periodic musical time, i.e., tempo, or beats.

We devise and implement an interface in the form of a handheld pendulum device. By swinging the pendulum, the participant is able to manipulate musical time, while the physical properties slightly guides them through its tendency to keep a stable period and fall back to a predetermined one.

\section{Related work}

As we direct our design towards approachability for all levels of expertise, we look for intuitive universalities among human music, and selected to focus on the formation and perception of patterned time based on the abundant argument that humans naturally tend to perceive time, and especially in the musical context, in cycles~\cite{Brower:Cog, Neisser_1976, Ravignani2016}. To paraphrase our forerunners, repeating cycles of time set the boundaries of our activities as well as form grounds on which we anticipate our sensory stimuli.

Indeed, there have been a number of devices that establish the link between periodic motion and stable temporal patterns.

\section{Design}

\section{Implementation}

\section{Discussion and Future Work}

\section{Conclusion}

\section{Acknowledgments}

\bibliographystyle{abbrv}
\bibliography{nime-references} 

\appendix
\section{Details of the Signal Processing Algorithms}
\subsection{Calibration}
\subsubsection{Accelerometer}
\subsubsection{Magnetometer}
\subsection{Movement Estimation}

\end{document}
